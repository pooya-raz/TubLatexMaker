
    \documentclass{article}
    \usepackage{fontspec,lipsum}
    \defaultfontfeatures{Ligatures=TeX}
    \usepackage[small,sf,bf]{titlesec}
    \setromanfont{Gentium Plus}
    \newfontfamily\arabicfont[Script=Arabic]{Amiri}
    \usepackage{polyglossia}
    \setmainlanguage{english}
    \setotherlanguage{arabic}
    \title{Twelver Usul Bibliography}
    \author{The TUB Team}
    \date{\today} 
    \begin{document}
    \maketitle
    \tableofcontents
    \pagebreak
    \section{Monographs with commentaries}
\begin{enumerate}
      \item \textbf{al-Dharīʿa ilā uṣūl al-sharīʿa}
        \newline
        \textarabic{الذريعة إلى أصول الشريعة}
        \newline
        al-Sharīf al-Murtaḍā
        \newline
        (436/1044)
        \newline
        \newline
        \textbf{Description}
        \newline	
        no data
        \newline
        \newline
    \textbf{Principle Manuscripts}
\begin{itemize}
    \item University of California, Los Angeles, Los Angeles (\#MS164), dated 5th-9th centuries/11th-15th centuries
    
    \item Majlis, Tehran (\#3185), dated 969/1561
    
    \item Majlis, Tehran (\#3794), dated 1025/1616
    
    \item Marʿashī, Qum (\#6519), dated 1027/1618
    
    \item Maktabat Amīr al-muʾminīn, Najaf (\#140), dated 1042/1632-33
    
    \item Gawharshād Library, Mashhad (\#196), dated 1043/1634
    
    \item Kāshif al-Ghiṭāʾ, Najaf (\#1661), dated 10 Ṣafar 1097/5 January 1686
    \end{itemize}
\textbf{Editions}
\begin{itemize}
        \item \emph{al-Dharīʿa ilā uṣūl al-sharīʿa} (ed. Abū l-Qāsim al-Gurjī), Modern print, Intishārāt-i Dāneshgāh-i Tehrān, Tehran, no data/1967
        
        \item \emph{al-Dharīʿa ilā uṣūl al-sharīʿa} (ed. Abū l-Qāsim al-Gurjī), Modern print, Intishārāt-i Dāneshgāh-i Tehrān, Tehran, no data/1985
        
        \item \emph{al-Dharīʿa ilā uṣūl al-sharīʿa} (ed. Abū l-Qāsim al-Gurjī), Modern print, Intishārāt-i Dāneshgāh-i Tehrān, Tehran, no data/1998
        
        \item \emph{al-Dharīʿa ilā uṣūl al-sharīʿa} (ed. al-lujnat al-ʿilmiyya fī Muʾassasat al-Imām al-Ṣādiq), Modern print, Muʾassasat al-Imām al-Ṣādiq, Qum, no data/2008
        
        \item \emph{al-Dharīʿa ilā uṣūl al-sharīʿa} (ed. al-Sayyid ʿAlī Riḍā al-Madadī), Modern print, Bunyād-i Pazuhishha-ye Islāmī – Āstān-i Quds-i Raḍawī, Mashhad, no data/2020
        \end{itemize}
\textbf{Commentaries}
\begin{enumerate}
      \item \textbf{Sharḥ al-Dharīʿa}
        \newline
        \textarabic{شرح الذريعه}
        \newline
        Kamāl al-Din al-Murtaḍā b. al-Muntahā b. al-Ḥusayn b. ʿAlī al-Ḥusaynī al-Marʿashī
        \newline
        (after 525/after 1130)
        \newline
        \newline
        \textbf{Description}
        \newline	
        no data
        \newline
        \newline
    \textbf{Principle Manuscripts}
\newline
no data\newline\textbf{Editions}
\newline
no data\newline
      \item \textbf{al-Mustaqṣā fī sharḥ al-Dharīʿa}
        \newline
        \textarabic{المستقصى في شرح الذريعة}
        \newline
        Quṭb al-Dīn al-Rāwandī
        \newline
        (573/1177)
        \newline
        \newline
        \textbf{Description}
        \newline	
        no data
        \newline
        \newline
    \textbf{Principle Manuscripts}
\newline
no data\newline\textbf{Editions}
\newline
no data\newline
      \item \textbf{al-Iʿtibār ʿalā kitāb al-Dharīʿa ilā uṣūl al-sharīʿa}
        \newline
        \textarabic{الاعتبار على كتاب الذريعة إلى أصول الشريعة}
        \newline
        Ibn Abī l-Ḥadīd
        \newline
        (655 or 656/1257 or 1258)
        \newline
        \newline
        \textbf{Description}
        \newline	
        no data
        \newline
        \newline
    \textbf{Principle Manuscripts}
\newline
no data\newline\textbf{Editions}
\newline
no data\newline
\end{enumerate}

      \item \textbf{al-ʿUdda fī uṣūl al-fiqh/ʿUddat al-uṣūl}
        \newline
        \textarabic{العدة في أصول الفقه - عدة الأصول}
        \newline
        al-Shaykh al-Ṭūsī
        \newline
        (460/1067)
        \newline
        \newline
        \textbf{Description}
        \newline	
        no data
        \newline
        \newline
    \textbf{Principle Manuscripts}
\newline
no data\newline\textbf{Editions}
\begin{itemize}
        \item \emph{al-ʿUdda fī uṣūl al-fiqh/ʿUddat al-uṣūl} (ed. no data), Lithograph, n.p., n.p., 1318/1896-7
        
        \item \emph{al-ʿUdda fī uṣūl al-fiqh/ʿUddat al-uṣūl} (ed. no data), Lithograph, Dār al-Khilāfa, Tehran, 1317/1899-90
        
        \item \emph{al-ʿUdda fī uṣūl al-fiqh/ʿUddat al-uṣūl} (ed. no data), Lithograph, no data, Bombay, 1318/1900-01
        
        \item \emph{ʿUddat al-uṣūl wa bi-dhaylihi al-Ḥāshiya al-Khalīliyya} (ed. Muḥammad Mahdī Ṭāhā Najaf), Modern print, Muʾassasat Āl al-Bayt, Qum, 1403/1982
        
        \item \emph{al-ʿUdda fī uṣūl al-fiqh/ʿUddat al-uṣūl} (ed. Muḥammad Mahdī Ṭāhā Najaf), Modern print, Sītāreh, Qum, 1417/1997
        
        \item \emph{al-ʿUdda fī uṣūl al-fiqh/ʿUddat al-uṣūl} (ed. unknown), Modern print, Muʾassasat al-Biʿthat, Qum, 1417/1997
        
        \item \emph{al-ʿUdda fī uṣūl al-fiqh/ʿUddat al-uṣūl} (ed. Muḥammad Riḍā al-Anṣārī al-Qummī), Modern print, Muḥammad Taqī ʿAlāqbandiyān, Qum, 1417/1997
        
        \item \emph{al-ʿUdda fī uṣūl al-fiqh/ʿUddat al-uṣūl} (ed. Muḥammad Riḍā al-Anṣārī al-Qummī), Modern print, n.p., Qum, no data/1997
        
        \item \emph{al-ʿUdda fī uṣūl al-fiqh/ʿUddat al-uṣūl} (ed. Muḥammad Riḍā al-Anṣārī al-Qummī), Modern print, Bustān-i Kitāb, Qum, 1432/2010
        \end{itemize}
\textbf{Commentaries}
\begin{itemize}
              \item \emph{ḤāshiyatʿUddat al-uṣūl}, Mullā Khalīl al-Qazwīnī (1089/1678)
            
              \item \emph{Najāt al-muslimīn wa-munjī al-hālikīn}, Muḥammad Mahdī b. Muḥammad Bāqir al-Ḥusaynī al-Mashhadī (11th century/17th century)
            
              \item \emph{Ḥāshiyat Sharḥ ʿUddat al-uṣūl}, ʿAlī Asghar b. Muḥammad Yūsuf al-Qazwīnī (1120/1709)
            
              \item \emph{Tanqīḥ al-marām fī ḥashiyat Sharḥ ʿUddat al-uṣūl}, ʿAlī Asghar b. Muḥammad Yūsuf al-Qazwīnī (1120/1709)
            
              \item \emph{Sharḥ ʿUddat al-uṣūl}, Muḥammad b. Muḥammad Ṣādiq al-Ḥusaynī al-Qazwīnī (12th century/18th century)
            \end{itemize}

      \item \textbf{Ghunyat al-nuzūʿ ilā ʿilmay al-uṣūl wa-l-furūʿ}
        \newline
        \textarabic{غنية النزوع إلى علمي الأصول والفروع}
        \newline
        Ibn Zuhra
        \newline
        (585/1189)
        \newline
        \newline
        \textbf{Description}
        \newline	
        no data
        \newline
        \newline
    \textbf{Principle Manuscripts}
\newline
no data\newline\textbf{Editions}
\begin{itemize}
        \item \emph{Muʿtaqad al-Imāmiyya} (ed. Muḥammad Taqī Dānish Pajū), Modern print, Chapkhāna-ye Danishgāh Tihrān, Tehran, no data/1961
        
        \item \emph{Ghunyat al-nuzūʿ ilā ʿilmay al-uṣūl wa-l-furūʿ} (ed. Ibrāhīm al-Bahādurī), Modern print, Muʾassasat al-Imām al-Ṣādiq, Qum, 1417/1994
        
        \item \emph{Ghunyat al-nuzūʿ ilā ʿilmay al-uṣūl wa-l-furūʿ} (ed. no data), Facsimile, Majlis-i Shūrā-i Islāmī, Tehran, 1390/2011/2012
        \end{itemize}

      \item \textbf{Maʿārij al-uṣūl}
        \newline
        \textarabic{معارج الأصول}
        \newline
        al-Muḥaqqiq al-Ḥillī
        \newline
        (676/1277)
        \newline
        \newline
        \textbf{Description}
        \newline	
        This work appeared as ''Nahj al-wuṣūl ilā maʿrifat al-uṣūl'' in some sources, particularly ''ijāza''s. Ṭihrānī (Dh 24:428-9\#2228) has suggested it is in fact the same as ''Maʿārij al-uṣūl'', noting that Mīr Damād called it ''Nahj al-Maʿārij''. Ṭihrānī also points out that a manuscript of it with the same title exists in Qom in the library of Shihāb l-Dīn l-Marʿashī l-Najafī (d. 1411/1990) and was written in Rajab 707AH. The editor of ''Maʿārij'' agrees, noting that the numbers and titles of chapters are similar (''Maʿārij al-uṣūl'', ed. Kashmīrī: 43-44).
        \newline
        \newline
    \textbf{Principle Manuscripts}
\newline
no data\newline\textbf{Editions}
\begin{itemize}
        \item \emph{Maʿārij al-uṣūl} (ed. no data), Lithograph, no data, Tehran, 1310/1892
        
        \item \emph{Maʿārij al-uṣūl} (ed. no data), Lithograph, no data, Tehran, 1310/1893
        
        \item \emph{Maʿārij al-uṣūl} (ed. Muḥammad Ḥusayn al-Raḍawī al-Kashmīrī), Modern print, Muʾassasat Āl al-Bayt, Qum, 1403/1982
        
        \item \emph{Maʿārij al-uṣūl} (ed. Muḥammad Ḥusayn al-Raḍawī al-Kashmīrī), Modern print, Muʾassasat al-Imām ʿAlī, London, no data/2003
        \end{itemize}
\textbf{Commentaries}
\begin{itemize}
              \item \emph{Sharḥ maʿārij al-uṣūl}, Baʿḍ al-aṣḥāb al-mutaʾakhkhirīn ʿan al-Shahīd al-Thānī (after 966/after 1558)
            
              \item \emph{Ḥāshiyah ʿalā Maʿārij al-uṣūl}, Mullā Muḥammad Amīn al-Astarābādī (1033 or 1036/1623 or 1626)
            
              \item \emph{Sharḥ maʿārij al-uṣūl}, Mīr Fayḍallāh al-Tafrīshī (1125/1616)
            
              \item \emph{Niẓām al-fuṣūl fī sharḥ nahj al-wuṣūl ilā maʿrifat al-uṣūl}, Fatḥ Allāh ibn ʿIlwān al-Qabbānī al-Dawraqī (1130/1718)
            
              \item \emph{Sharḥ Maʿārij al-uṣūl}, Mīr ʿAbd al-Ṣamad al-Ḥusaynī al-Hamadānī (1216/1802)
            \end{itemize}

      \item \textbf{Nahj al-wuṣūl ilā maʿrifat al-uṣūl}
        \newline
        \textarabic{نهج الأصول إلى معرفة الأصول}
        \newline
        al-Muḥaqqiq al-Ḥillī
        \newline
        (676/1277)
        \newline
        \newline
        \textbf{Description}
        \newline	
        no data
        \newline
        \newline
    \textbf{Principle Manuscripts}
\newline
no data\newline\textbf{Editions}
\newline
no data\newline\textbf{Commentaries}
\begin{itemize}
              \item \emph{Niẓām al-fuṣūl fī sharḥ nahj al-wuṣūl ilā maʿrifat al-uṣūl}, Fatḥallāh b. ʿUlwān al-Qupānī al-Dawraqī al-Kaʿbī (1130/1717)
            \end{itemize}
\end{enumerate}\section{Monographs without commentaries}
\begin{enumerate}
      \item \textbf{al-Burhān fī wujūh al-bayān}
        \newline
        \textarabic{البرهان في وجوه البيان}
        \newline
        Abū l-Ḥasan Isḥāq b. Ibrāhīm b. Sulaymān b. Wahab al-Kātib
        \newline
        (4th century/10th century)
        \newline
        \newline
        \textbf{Description}
        \newline	
        no data
        \newline
        \newline
    \textbf{Principle Manuscripts}
\newline
no data\newline\textbf{Editions}
\begin{itemize}
        \item \emph{al-Burhān fī wujūh al-bayān} (ed. no data), no data, no data, no data, no data/1930
        
        \item \emph{al-Burhān fī wujūh al-bayān} (ed. Ṭāhā Ḥusayn), Modern print, no data, no data, no data/1938
        
        \item \emph{al-Burhān fī wujūh al-bayān} (ed. Aḥmad Maṭlūb), Modern print, Jāmiʿat Baghdād, Baghdad, 1386/1967
        
        \item \emph{al-Burhān fī wujūh al-bayān} (ed. Ḥifnī Muḥammad Sharaf), Modern print, Maktabat al-Shabāb, Cairo, 1389/1969
        
        \item \emph{al-Burhān fī wujūh al-bayān} (ed. no data), Modern print, Maktabat al-Rushd Nāshirūn, Riyadh, no data/2012
        \end{itemize}

      \item \textbf{Kitāb Naqd al-nathr}
        \newline
        \textarabic{كتاب نقد النثر}
        \newline
        Abū l-Ḥasan Isḥāq b. Ibrāhīm b. Sulaymān b. Wahab al-Kātib
        \newline
        (4th century/10th century)
        \newline
        \newline
        \textbf{Description}
        \newline	
        no data
        \newline
        \newline
    \textbf{Principle Manuscripts}
\newline
no data\newline\textbf{Editions}
\begin{itemize}
        \item \emph{Kitāb Naqd al-nathr} (ed. Ṭāhā Ḥusayn), Modern print, Lajnat al-Taʼlīf wa-l-Tarjama wa-l-Nashr, Cairo, no data/1940
        
        \item \emph{Kitāb Naqd al-nathr} (ed. no data), Modern print, Dar al-Kutub al-ʿIlmiyya, Beirut, no data/1980
        
        \item \emph{Kitāb Naqd al-nathr} (ed. no data), Modern print, Dar al-Kutub al-ʿIlmiyya, Beirut, no data/1982
        
        \item \emph{Kitāb Naqd al-nathr} (ed. no data), Modern print, Dār al-Maʻārif lil-Ṭibāʻa wa-l-Nashr, Sousse, no data/2004
        \end{itemize}

      \item \textbf{Uṣūl al-fiqh wa dirāyat al-ḥadīth}
        \newline
        \textarabic{أصول الفقه ودراية الحديث}
        \newline
        al-Muḥaqqiq al-Ḥillī
        \newline
        (676/1277)
        \newline
        \newline
        \textbf{Description}
        \newline	
        no data
        \newline
        \newline
    \textbf{Principle Manuscripts}
\begin{itemize}
    \item KG, Najaf (\#1118/1), dated 1023/1614
    \end{itemize}
\textbf{Editions}
\newline
no data\newline
      \item \textbf{Nihāyat al-wuṣūl ilā ʿilm al-uṣūl}
        \newline
        \textarabic{نهاية الوصول إلى علم الأصول}
        \newline
        al-ʿAllāma al-Ḥillī
        \newline
        (726/1325)
        \newline
        \newline
        \textbf{Description}
        \newline	
        no data
        \newline
        \newline
    \textbf{Principle Manuscripts}
\newline
no data\newline\textbf{Editions}
\begin{itemize}
        \item \emph{Nihāyat al-wuṣūl ilā ʿilm al-uṣūl} (ed. Ibrāhīm al-Bahādurī), Modern print, Muʾassasat al-Imām al-Ṣādiq, Qum, 1425/2004
        
        \item \emph{Nihāyat al-wuṣūl ilā ʿilm al-uṣūl} (ed. Ibrāhīm al-Bahādurī), Modern print, Muʾassasat al-Imām al-Ṣādiq, Qum, 1427/2006
        
        \item \emph{Nihāyat al-wuṣūl ilā ʿilm al-uṣūl} (ed. unknown), Modern print, Muʾassasat Āl al-Bayt, Qum, 1431/2010
        \end{itemize}

      \item \textbf{Muntahā l-wuṣūl ilā ʿilmay al-kalām wa-l-uṣūl}
        \newline
        \textarabic{منتهى الوصول إلي علمي الكلام والأصول}
        \newline
        al-ʿAllāma al-Ḥillī
        \newline
        (726/1325)
        \newline
        \newline
        \textbf{Description}
        \newline	
        no data
        \newline
        \newline
    \textbf{Principle Manuscripts}
\newline
no data\newline\textbf{Editions}
\newline
no data\newline\end{enumerate}\section{Gloss (ḥāshīyah)}
\begin{enumerate}
      \item \textbf{Ḥāshiyat Ḥāshiyat Sharḥ Mukhtaṣar al-uṣūl}
        \newline
        \textarabic{حاشية حاشية شرح مختصر الأصول}
        \newline
        ʿAlī b. Muḥammad al-Ṭūsī
        \newline
        (c. 877/c. 1472)
        \newline
        \newline
        \textbf{Description}
        \newline	
        no data
        \newline
        \newline
    \textbf{Principle Manuscripts}
\newline
no data\newline\textbf{Editions}
\newline
no data\newline
      \item \textbf{Ḥāshiyat al-Talwīḥ ilā kashf ḥaqāʾiq al-tanqīḥ}
        \newline
        \textarabic{حاشية التلويح إلي كشف حقائق التنقيح}
        \newline
        ʿAlī b. Muḥammad al-Ṭūsī
        \newline
        (c. 877/c. 1472)
        \newline
        \newline
        \textbf{Description}
        \newline	
        no data
        \newline
        \newline
    \textbf{Principle Manuscripts}
\newline
no data\newline\textbf{Editions}
\newline
no data\newline
      \item \textbf{Ḥāshiyat Ḥāshiyat Mukhtaṣar al-uṣūl}
        \newline
        \textarabic{حاشية حاشية مختصر الأصول}
        \newline
        Mīr Ṣadr al-Dīn Muḥammad al-Dashtakī al-Shīrāzī
        \newline
        (903/1497-98)
        \newline
        \newline
        \textbf{Description}
        \newline	
        no data
        \newline
        \newline
    \textbf{Principle Manuscripts}
\newline
no data\newline\textbf{Editions}
\newline
no data\newline
      \item \textbf{Ḥāshiyat Ḥāshiyat Sharḥ Mukhtaṣar al-uṣūl - al-Dawwānī}
        \newline
        \textarabic{حاشية حاشية شرح مختصر الأصول - الدواني}
        \newline
        Muḥammad b. Asʿad al-Dawwānī
        \newline
        (c. 908/c. 1502)
        \newline
        \newline
        \textbf{Description}
        \newline	
        no data
        \newline
        \newline
    \textbf{Principle Manuscripts}
\newline
no data\newline\textbf{Editions}
\newline
no data\newline
      \item \textbf{Ḥāshiyat ʿalā Sharḥ Mukhtaṣar al-uṣūl al-ʿAḍudī}
        \newline
        \textarabic{حاشية على شرح مختصر الأصول العضدي}
        \newline
        al-Muqaddas al-Ardabīlī
        \newline
        (993/1585)
        \newline
        \newline
        \textbf{Description}
        \newline	
        no data
        \newline
        \newline
    \textbf{Principle Manuscripts}
\newline
no data\newline\textbf{Editions}
\newline
no data\newline\end{enumerate}\section{Marginal notes (taʿlīqa)}
\begin{enumerate}
      \item \textbf{Taʿlīqāt ʿala Ḥāshiyat Sharḥ Mukhtaṣar}
        \newline
        \textarabic{تعليقات علي حاشية شرح المختصر}
        \newline
        Aḥmad b. Zayn al-Dīn al-ʿAlawī al-ʿĀmilī
        \newline
        (1054/1644-45)
        \newline
        \newline
        \textbf{Description}
        \newline	
        no data
        \newline
        \newline
    \textbf{Principle Manuscripts}
\newline
no data\newline\textbf{Editions}
\newline
no data\newline
      \item \textbf{Maʿālim al-uṣūl maʿa taʿlīqāt Sulṭān al-ʿUlamāʾ}
        \newline
        \textarabic{معالم الأصول مع تعليقات سلطان العلماء}
        \newline
        Sulṭān al-ʿUlamāʾ
        \newline
        (1064/1653)
        \newline
        \newline
        \textbf{Description}
        \newline	
        no data
        \newline
        \newline
    \textbf{Principle Manuscripts}
\newline
no data\newline\textbf{Editions}
\begin{itemize}
        \item \emph{Maʿālim al-uṣūl} (ed. ʿAlī Muḥammadī), Modern print, Dār al-Fikr, Qum, no data/1995
        
        \item \emph{Maʿālim al-uṣūl} (ed. ʿAlī Muḥammadī), Modern print, Dār al-Fikr, Qum, no data/1997
        \end{itemize}

      \item \textbf{Taʿlīqāt ʿalā al-Zubda}
        \newline
        \textarabic{تعليقات علي الزبدة}
        \newline
        Muḥammad Taqī b. ʿAbd al-Ḥusayn al-Naṣīrī al-Ṭūsī
        \newline
        (11th century/17th century)
        \newline
        \newline
        \textbf{Description}
        \newline	
        no data
        \newline
        \newline
    \textbf{Principle Manuscripts}
\newline
no data\newline\textbf{Editions}
\newline
no data\newline
      \item \textbf{al-Tajazzī - Taʿlīqa ʿalā masʾalat tajazzī l-ijtihād}
        \newline
        \textarabic{التجزي - تعليقة على مسئلة تجزي الاجتهاد}
        \newline
        Mīr ʿAbd al-Ṣamad al-Ḥusaynī al-Hamadānī
        \newline
        (1216/1802)
        \newline
        \newline
        \textbf{Description}
        \newline	
        no data
        \newline
        \newline
    \textbf{Principle Manuscripts}
\newline
no data\newline\textbf{Editions}
\newline
no data\newline
      \item \textbf{Taʿlīqa mutaʿaliqqa bi-l-naskh}
        \newline
        \textarabic{تعليقة متعلقة بالنسخ}
        \newline
        Sharīf al-ʿulamāʾ al-Māzandarānī
        \newline
        (1245 or 1246/1829 or 1830)
        \newline
        \newline
        \textbf{Description}
        \newline	
        no data
        \newline
        \newline
    \textbf{Principle Manuscripts}
\newline
no data\newline\textbf{Editions}
\newline
no data\newline\end{enumerate}\section{Treatise (risāla)}
\begin{enumerate}
      \item \textbf{Jawābāt al-masāʾil al-Muṣiliyāt al-thālitha}
        \newline
        \textarabic{جوابات المسائل الموصليات الثالثة}
        \newline
        al-Sharīf al-Murtaḍā
        \newline
        (436/1044)
        \newline
        \newline
        \textbf{Description}
        \newline	
        no data
        \newline
        \newline
    \textbf{Principle Manuscripts}
\newline
no data\newline\textbf{Editions}
\begin{itemize}
        \item \emph{Jawābāt al-masāʾil al-Muṣiliyāt al-thālitha} (ed. al-Sayyid Mahdī Rajāʾī), Modern print, Dār al-Qurʾān al-Karīm, Qum, 1405/1984
        \end{itemize}

      \item \textbf{Uṣūl al-fiqh (Kitāb fī)}
        \newline
        \textarabic{أصول الفقه (كتاب في)}
        \newline
        al-Sharīf al-Murtaḍā
        \newline
        (436/1044)
        \newline
        \newline
        \textbf{Description}
        \newline	
        no data
        \newline
        \newline
    \textbf{Principle Manuscripts}
\newline
no data\newline\textbf{Editions}
\newline
no data\newline
      \item \textbf{Masʾalat ʿadam takhṭiʾa al-ʿāmil bi-khabar al-wāḥid}
        \newline
        \textarabic{مسألة عدم تخطئة العامل بخبر الواحد}
        \newline
        al-Sharīf al-Murtaḍā
        \newline
        (436/1044)
        \newline
        \newline
        \textbf{Description}
        \newline	
        no data
        \newline
        \newline
    \textbf{Principle Manuscripts}
\newline
no data\newline\textbf{Editions}
\begin{itemize}
        \item \emph{Masʾalat ʿadam takhṭiʾa al-ʿāmil bi-khabar al-wāḥid} (ed. al-Sayyid Mahdī Rajāʾī), Modern print, Dār al-Qurʾān al-Karīm, Qum, 1405/1984
        \end{itemize}

      \item \textbf{Masʾalat fī nafy al-ḥukm bi ʿadam al-dalīl ʿalayhi}
        \newline
        \textarabic{مسألة في نفي الحكم بعدم الدليل عليه}
        \newline
        al-Sharīf al-Murtaḍā
        \newline
        (436/1044)
        \newline
        \newline
        \textbf{Description}
        \newline	
        no data
        \newline
        \newline
    \textbf{Principle Manuscripts}
\newline
no data\newline\textbf{Editions}
\begin{itemize}
        \item \emph{Masʾalat fī nafy al-ḥukm bi ʿadam al-dalīl ʿalayhi} (ed. al-Sayyid Mahdī Rajāʾī), Modern print, Dār al-Qurʾān al-Karīm, Qum, 1405/1984
        \end{itemize}

      \item \textbf{Muqaddimat al-wājib (Risālat fī)}
        \newline
        \textarabic{مقدمة الواجب (رسالة في)}
        \newline
        al-Sharīf al-Murtaḍā
        \newline
        (436/1044)
        \newline
        \newline
        \textbf{Description}
        \newline	
        no data
        \newline
        \newline
    \textbf{Principle Manuscripts}
\newline
no data\newline\textbf{Editions}
\newline
no data\newline\end{enumerate}\section{Summary (khulāṣa/mukhtaṣar)}
\begin{enumerate}
      \item \textbf{(Mukhtaṣar) al-Tadhkira bi-uṣul al-fiqh}
        \newline
        \textarabic{مختصر) التذكرة بأصول الفقه)}
        \newline
        al-Karājukī/al-Karājakī al-Ṭarāblūsī
        \newline
        (449/1057)
        \newline
        \newline
        \textbf{Description}
        \newline	
        This is a summary of al-Tadhkira bi-uṣūl al-fiqh.
        \newline
        \newline
    \textbf{Principle Manuscripts}
\newline
no data\newline\textbf{Editions}
\begin{itemize}
        \item \emph{(Mukhtaṣar) al-Tadhkira bi-uṣul al-fiqh} (ed. no data), Lithograph, no data, Tabriz, 1322/1904
        
        \item \emph{(Mukhtaṣar) al-Tadhkira bi-uṣul al-fiqh} (ed. Mahdī Najaf), Modern print, Kungre-ye Shaykh Mufīd, Qum, 1413/1992
        
        \item \emph{(Mukhtaṣar) al-Tadhkira bi-uṣul al-fiqh} (ed. Mahdī Najaf), Modern print, Dār al-Mufīd li-Ṭibāʿa wa-l-Nashr wa-l-Tawzīʿ, Beirut, 1414/1993
        \end{itemize}

      \item \textbf{Uṣūl al-fiqh (al-Adilla ʿalā mashrūʿiyyat al-ʿamal bi-l-ẓunūn)}
        \newline
        \textarabic{أصول الفقه (الأدلة على مشروعية العمل بالظنون)}
        \newline
        Fakhr al-Dīn al-Ṭurayḥī
        \newline
        (1085 or 1087/1674 or 1676)
        \newline
        \newline
        \textbf{Description}
        \newline	
        no data
        \newline
        \newline
    \textbf{Principle Manuscripts}
\newline
no data\newline\textbf{Editions}
\newline
no data\newline
      \item \textbf{Manāhil al-shawārid fī talkhīṣ al-Maqāṣid}
        \newline
        \textarabic{مناهل الشوارد في تلخيص المقاصد}
        \newline
        Mullā ʿAlī al-Ārānī al-Kāshānī
        \newline
        (after 1198/after 1783-84)
        \newline
        \newline
        \textbf{Description}
        \newline	
        no data
        \newline
        \newline
    \textbf{Principle Manuscripts}
\begin{itemize}
    \item Masjid-i Aʿẓam, Qum (\#2543), dated 1224/1809
    
    \item Marʿashī, Qum (\#13164), dated 1225/1810
    \end{itemize}
\textbf{Editions}
\newline
no data\newline
      \item \textbf{Hidāyat al-abrār (muntakhab)}
        \newline
        \textarabic{هداية الأبرار (منتخب)}
        \newline
        Mīrzā Muḥammad al-Akhbārī
        \newline
        (1232 or 1233/1817 or 1818)
        \newline
        \newline
        \textbf{Description}
        \newline	
        no data
        \newline
        \newline
    \textbf{Principle Manuscripts}
\newline
no data\newline\textbf{Editions}
\newline
no data\newline
      \item \textbf{Mulakhkhaṣ al-fawāʾid al-saniyya wa muntakhab al-farāʾid al-Ḥusayniyya}
        \newline
        \textarabic{ملخص الفوائد السنية ومنتخب الفرائد الحسينية}
        \newline
        Muḥammad Ḥasan b. Muḥammad Maʿṣūm al-Qazwīnī
        \newline
        (1240/1824-25)
        \newline
        \newline
        \textbf{Description}
        \newline	
        no data
        \newline
        \newline
    \textbf{Principle Manuscripts}
\newline
no data\newline\textbf{Editions}
\newline
no data\newline\end{enumerate}\section{Poem (manẓūma)}
\begin{enumerate}
      \item \textbf{Naẓm al-uṣūl}
        \newline
        \textarabic{نظم الأصول}
        \newline
        al-Shaykh al-Bahāʾī
        \newline
        (1030 or 1031/1620 or 1621)
        \newline
        \newline
        \textbf{Description}
        \newline	
        no data
        \newline
        \newline
    \textbf{Principle Manuscripts}
\begin{itemize}
    \item AQR, Mashhad (\#27588), dated 1226/1811
    \end{itemize}
\textbf{Editions}
\newline
no data\newline
      \item \textbf{Irshād al-ṭālib ilā manẓūmat al-kawākib}
        \newline
        \textarabic{إرشاد الطالب إلى منظومة الكواكب}
        \newline
        Muḥammad b. Ḥasan al-Kawākibī
        \newline
        (1096/1684-85)
        \newline
        \newline
        \textbf{Description}
        \newline	
        no data
        \newline
        \newline
    \textbf{Principle Manuscripts}
\newline
no data\newline\textbf{Editions}
\begin{itemize}
        \item \emph{Irshād al-ṭālib ilā manẓūmat al-kawākib} (ed. no data), no data, no data, Bulaq, 1322/1904
        \end{itemize}

      \item \textbf{Naẓm Zubdat al-uṣūl - al-Ḥusaynī al-Sayfī}
        \newline
        \textarabic{نظم زبدة الأصول - الحسيني السيفي}
        \newline
        al-Mīrzā Qiwām al-Dīn
        \newline
        (alive in 1104/alive in 1693)
        \newline
        \newline
        \textbf{Description}
        \newline	
        This work is dedicated to Muqarrib al-Khāqān Ismāʿīl. It contains 1001 couplets (bayt).
        \newline
        \newline
    \textbf{Principle Manuscripts}
\begin{itemize}
    \item Ḥawza-ye Āstān-i Ḥaḍrat ʿAbd al-Aẓīm, Tehran (\#53), dated 1108/1697
    
    \item AQR, Mashhad (\#2953), dated 1129/1717
    
    \item Farhād Muʿtamad, Tehran (\#50), dated 12th century/18th century
    
    \item Millī, Tehran (\#2076), dated 12th century/18th century
    
    \item Gawharshād, Mashhad (\#1049/8), dated 13th century/19th century
    \end{itemize}
\textbf{Editions}
\newline
no data\newline
      \item \textbf{Uṣūl al-fiqh}
        \newline
        \textarabic{أصول الفقه}
        \newline
        ʿAlī b. al-Ḥusayn al-ʿĀmilī
        \newline
        (1135/1722-23)
        \newline
        \newline
        \textbf{Description}
        \newline	
        no data
        \newline
        \newline
    \textbf{Principle Manuscripts}
\begin{itemize}
    \item Majlis, Tehran (\#2/400), dated 1281/1864-65
    \end{itemize}
\textbf{Editions}
\newline
no data\newline
      \item \textbf{Wasīlat al-wuṣūl}
        \newline
        \textarabic{وسيلة الوصول}
        \newline
        ʿAlī b. al-Ḥusayn al-ʿĀmilī b. Muḥyī l-Dīn b. ʿAbd al-Laṭīf al-Jāmiʿī al-ʿĀmilī
        \newline
        (1135/1723)
        \newline
        \newline
        \textbf{Description}
        \newline	
        no data
        \newline
        \newline
    \textbf{Principle Manuscripts}
\newline
no data\newline\textbf{Editions}
\newline
no data\newline\end{enumerate}\section{Refutation (radd)}
\begin{enumerate}
      \item \textbf{Nuṣrat al-aṣḥāb}
        \newline
        \textarabic{نصرة الأصحاب}
        \newline
        ʿAlī Naqī b. Muḥammad Hāshim al-Ṭughāʾī al-Kamarehʾī
        \newline
        (c. 1060/c. 1650)
        \newline
        \newline
        \textbf{Description}
        \newline	
        no data
        \newline
        \newline
    \textbf{Principle Manuscripts}
\newline
no data\newline\textbf{Editions}
\newline
no data\newline
      \item \textbf{Shubhat al-dawr (al-Radd ʿalā Mullā Khalīl al-Qazwīnī)}
        \newline
        \textarabic{شبهة الدور (الرد على ملا خليل القزويني)}
        \newline
        al-Muḥaqqiq al-Khwānsārī
        \newline
        (1099/1687-88)
        \newline
        \newline
        \textbf{Description}
        \newline	
        no data
        \newline
        \newline
    \textbf{Principle Manuscripts}
\newline
no data\newline\textbf{Editions}
\newline
no data\newline
      \item \textbf{Radd Shubhat al-dawr}
        \newline
        \textarabic{رد شبهة الدور}
        \newline
        Muḥammad Ḥusayn b. Muḥammad Bāqir al-Yazdī
        \newline
        (11th century/17th century)
        \newline
        \newline
        \textbf{Description}
        \newline	
        no data
        \newline
        \newline
    \textbf{Principle Manuscripts}
\newline
no data\newline\textbf{Editions}
\newline
no data\newline
      \item \textbf{Taqlīd al-mujtahid}
        \newline
        \textarabic{تقليد المجتهد}
        \newline
        Muḥammad ʿAlī b. Muḥammad Ṣādiq al-Nayshāburī
        \newline
        (12th century/18th century)
        \newline
        \newline
        \textbf{Description}
        \newline	
        no data
        \newline
        \newline
    \textbf{Principle Manuscripts}
\newline
no data\newline\textbf{Editions}
\newline
no data\newline
      \item \textbf{Radd ʿalā al-qawl bi ḥujjiyat al-ẓann}
        \newline
        \textarabic{رد علی القول بحجة الظن}
        \newline
        al-Muḥaqqiq al-Kāẓimī
        \newline
        (1227/1812)
        \newline
        \newline
        \textbf{Description}
        \newline	
        no data
        \newline
        \newline
    \textbf{Principle Manuscripts}
\newline
no data\newline\textbf{Editions}
\newline
no data\newline\end{enumerate}\section{Taqrīrāt}
\begin{enumerate}
      \item \textbf{Taqrīrāt al-fiqh wa-l-uṣūl}
        \newline
        \textarabic{تقريرات الفقه والأصول}
        \newline
        ʿAlī Riḍā
        \newline
        (12th century/18th century)
        \newline
        \newline
        \textbf{Description}
        \newline	
        no data
        \newline
        \newline
    \textbf{Principle Manuscripts}
\newline
no data\newline\textbf{Editions}
\newline
no data\newline
      \item \textbf{al-Qatʿ wa-l-ẓann}
        \newline
        \textarabic{القطع والظن}
        \newline
        Aḥmad b. Muḥammad Muḥsin
        \newline
        (13th Century/18th/19th Century)
        \newline
        \newline
        \textbf{Description}
        \newline	
        Taqrīrāt of [[Shaykh Anṣārī]]
        \newline
        \newline
    \textbf{Principle Manuscripts}
\begin{itemize}
    \item Fāḍil Khawnsārī, Khwansar (\#214), dated 13th century/19th century
    \end{itemize}
\textbf{Editions}
\newline
no data\newline
      \item \textbf{al-Mawāʾid al-uṣūliyya fī al-ghurafāt al-gharwiyya}
        \newline
        \textarabic{الموائد الاصولية في الغرفات الغروية}
        \newline
        Sayyid Muḥammad
        \newline
        (13th Century/18th/19th Century)
        \newline
        \newline
        \textbf{Description}
        \newline	
        no data
        \newline
        \newline
    \textbf{Principle Manuscripts}
\begin{itemize}
    \item Markaz-i Aḥyāʾ, Qum (\#2998), dated 13th Century/18th/19th Century
    
    \item Markaz-i Aḥyāʾ, Qum (\#2165), dated 1260/1844
    \end{itemize}
\textbf{Editions}
\newline
no data\newline
      \item \textbf{Uṣūl al-fiqh - al-Shūshtarī}
        \newline
        \textarabic{أصول الفقه - الشوشتري}
        \newline
        ʿAlī al-Shūshtarī
        \newline
        (1217/1802-3)
        \newline
        \newline
        \textbf{Description}
        \newline	
        no data
        \newline
        \newline
    \textbf{Principle Manuscripts}
\newline
no data\newline\textbf{Editions}
\newline
no data\newline
      \item \textbf{Taqrīrāt al-uṣūl}
        \newline
        \textarabic{تقريرات الأصول}
        \newline
        ʿAlī Aṣghar al-Yazdī
        \newline
        (1240/1824-5)
        \newline
        \newline
        \textbf{Description}
        \newline	
        no data
        \newline
        \newline
    \textbf{Principle Manuscripts}
\newline
no data\newline\textbf{Editions}
\newline
no data\newline\end{enumerate}\section{ Translation}
\begin{enumerate}
      \item \textbf{Ijtihād wa taqlīd - al-Ḥusaynī al-Ṭabāṭabāʾī}
        \newline
        \textarabic{اجتهاد و تقليد - الحسيني الطباطبائي}
        \newline
        Maḥmūd al-Ḥusaynī al-Ṭabāṭabāʾī
        \newline
        (11th century/17th century)
        \newline
        \newline
        \textbf{Description}
        \newline	
        no data
        \newline
        \newline
    \textbf{Principle Manuscripts}
\newline
no data\newline\textbf{Editions}
\newline
no data\newline
      \item \textbf{Maʿālim al-uṣūl (translation)}
        \newline
        \textarabic{معالم الأصول (ترجمة)}
        \newline
        Muḥammad Hādī b. Muḥammad Ṣāliḥ al-Māzandarānī
        \newline
        (1120/1708-09)
        \newline
        \newline
        \textbf{Description}
        \newline	
        no data
        \newline
        \newline
    \textbf{Principle Manuscripts}
\begin{itemize}
    \item Gulpāygānī, Qum (\#6/73-983), dated 11th century/17th century
    
    \item Markaz Iḥyāʾ, Qum (\#3935), dated 1149/1736
    
    \item Saryazdī, Yazd (\#173/2), dated 13th century/19th century
    \end{itemize}
\textbf{Editions}
\newline
no data\newline\end{enumerate}
    \end{document}
    